%%%%%%%%%%%%%%%%%%%%%%%%%%%%%%%%%%%%%%%%%%%%%%%%%%%%%%%%%%%%%%%%%%%%%%%%%%%%%
% Chapter 4: Conclusiones y Trabajos Futuros 
%%%%%%%%%%%%%%%%%%%%%%%%%%%%%%%%%%%%%%%%%%%%%%%%%%%%%%%%%%%%%%%%%%%%%%%%%%%%%%%

Para la conclusión trataremos dos punos de enfoque de nuestro trabajo, es decir, por un lado el aprendizaje teórico de las matemáticas y por otro lado su aplicación computacional.
%%%%%%%%%%%%%%%%%%%%%%%%%%%%%%%%%%%%%%%%%%%%%%%%%%%%%%%%%%%%%
\section{Conclusiones de los aspectos teóricos}
\label{4.1}
Hasta ahora, en lo que a teoría matemática y más concretamente a aproximación de funciones se refiere, solo habíamos tratado con aproximaciones en torno a un punto haciendo uso del polinomio de Taylor y acotado funciones en determinados intervalos haciendo uso de teoremas de continuidad pero nunca sin saber cual sería realmente dicha función ya que no se trataba de aproximarlas. Además se ha hecho uso del cálculo de diferencias divididas, que también es otro concepto que antes de la realización de este trabajo no nos era familiar. En general, la realización de este trabajo nos ha sido muy útil pues hemos obtenido nuevos conocimientos matemáticos de gran utilidad y considerados de un mayor nivel del que nos corresponde.



%%%%%%%%%%%%%%%%%%%%%%%%%%%%%%%%%%%%%%%%%%%%%%%%%%%%%%%%%%%%%%
\section{Conclusiones del aspecto computacional}
\label{4.2}
Desde el punto de vista informático, esta tarea nos ha servido para adquirir destrezas en la programación en Python, el uso del sistema operativo Bardinux, diferentes editores de texto y en especial en la codificación en \LaTeX y en Beamer de artículos y presentaciones. Esto nos podrá ayudar en un futuro a realizar informes y presentaciones de manera mas formal y profesional.
