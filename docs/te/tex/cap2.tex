%%%%%%%%%%%%%%%%%%%%%%%%%%%%%%%%%%%%%%%%%%%%%%%%%%%%%%%%%%%%%%%%%%%%%%%%%%%%%%%
% Chapter 2: Fundamentos Teóricos 
%%%%%%%%%%%%%%%%%%%%%%%%%%%%%%%%%%%%%%%%%%%%%%%%%%%%%%%%%%%%%%%%%%%%%%%%%%%%%%%

%++++++++++++++++++++++++++++++++++++++++++++++++++++++++++++++++++++++++++++++


%++++++++++++++++++++++++++++++++++++++++++++++++++++++++++++++++++++++++++++++

\section{Interpolación Polinómica}
\label{2:sec:1}
En análisis numérico, la interpolación polinomial es una técnica de interpolación de un conjunto de datos o de una función por un polinomio. Es decir, asumimos que sólo se conoce la imagen de una función en un número finito de abscisas. En muchos de los casos, ni siquiera se conocerá la expresión de la función.\par El objetivo de esta técnica es el de hallar un polinomio que tome los valores antes mencionados y que permita hallar aproximaciones de valores desconocidos para la función. Para segurar la precisión del polinomio se dispondrá de una fórmula del error de interpolación que permitirá ajustarlo.

\section{Cálculo del polinomio interpolador de Newton}
\label{2:sec:2}
  Existen varios métodos generales de interpolación polinómica que permiten aproximar una función por medio de un polinomio de grado m. En este informe, se recogerá exclusivamente el método de las diferencias divididas de Newton.\par Definición: Sea $f_{n}$  una variable discreta de n  elementos y sea $x_{n}$ otra variable discreta de n elementos los cuales corresponden a la imagen y la abcisa de los datos que se quieran interpolar:
  \[f(x_{k})=f_{k}, \hspace{0.5 cm} k=1,...,n\]\par
  Una gran ventaja sobre la forma clásica de Lagrange es que podemos agregar un mayor número de nodos a la tabla de datos, lo que nos facilitará en gran medida el cálculo del polinomio, sobretodo, en aquellos casos en los que el grado del polinomio que se quiere calcular es bastante elevado.\par Pongamos un ejemplo: el polinomio de grado n-1  resultante de aplicar este método, tendrá la forma:
  \[\sum_{i=0}^{n-1} {a_j}{g_j}(x)\]\par Donde:\par \[ g_{j}=\prod_{j=1}^{j-1}{(x-x_{i})}\]\par
  \[ a_{j}=f[x_{0},x_{1},...,x_{j-1},x_{j},]\]\par Los coeficientes $a_{j}$ son las llamadas diferencias divididas. El cálculo de estas diferencias es el paso que caracteriza este método. Tenemos que las diferencias divididas serían de la forma siguiente:
  
  \vspace{1.5 true cm}
  %--------------------------------------------------------------------------
\begin{table}[!ht]
\begin{center}
\begin{tabular}{|c|c|} \hline 
\textbf{Tiempo  } & \textbf{Velocidad} \\ 
\textbf{($\pm$ 0.001 s)} & \textbf{($\pm$ 0.1 m/s)} \\ \hline \hline
1.234 &
67.8
\\
\hline

2.345 &
78.9
\\
\hline

3.456 &
89.1
\\
\hline

4.567 &
91.2
\\
\hline

\end{tabular}
\end{center}
\caption{Resultados experimentales de tiempo (s) y velocidad (m/s)}
\label{tab:1}
\end{table}


