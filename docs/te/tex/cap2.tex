%%%%%%%%%%%%%%%%%%%%%%%%%%%%%%%%%%%%%%%%%%%%%%%%%%%%%%%%%%%%%%%%%%%%%%%%%%%%%%%
% Chapter 2: Fundamentos Teóricos 
%%%%%%%%%%%%%%%%%%%%%%%%%%%%%%%%%%%%%%%%%%%%%%%%%%%%%%%%%%%%%%%%%%%%%%%%%%%%%%%

%++++++++++++++++++++++++++++++++++++++++++++++++++++++++++++++++++++++++++++++

En este capítulo se han de presentar los antecedentes teóricos y prácticos que
apoyan el tema objeto de la investigación.

%++++++++++++++++++++++++++++++++++++++++++++++++++++++++++++++++++++++++++++++

\section{Primer apartado del segundo capítulo}
\label{2:sec:1}
  Primer párrafo de la primera sección.

\section{Segundo apartado del segundo capítulo}
\label{2:sec:2}
  Primer párrafo de la segunda sección.

  En \LaTeX{}~\cite{Lamport:LDP94} es sencillo escribir expresiones
  matemáticas como $a=\sum_{i=1}^{10} {x_i}^{3}$
  y deben ser escritas entre dos simbolos \$.
  Los superindices se obtienen con el simbolo \^{}, y
  los subindices con el simbolo \_.
  Por ejemplo: $x^2 \times y^{\alpha + \beta}$.
  También se puede escribir fórmulas centradas:
  \[h^2=a^2+b^2\]