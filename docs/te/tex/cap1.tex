 %%%%%%%%%%%%%%%%%%%%%%%%%%%%%%%%%%%%%%%%%%%%%%%%%%%%%%%%%%%%%%%%%%%%%%%%%%%%%
% Chapter 1: Motivación y Objetivos
%%%%%%%%%%%%%%%%%%%%%%%%%%%%%%%%%%%%%%%%%%%%%%%%%%%%%%%%%%%%%%%%%%%%%%%%%%%%%%%

Los objetivos para los que se plantea este trabajo, son el adquirir conocimientos y mejorar nuestras habilidades en el uso del lenguaje de programanción PYTHON, procesador de texto \LaTeX{} y una clase de \LaTeX{} que nos permite diseñar presentaciones, BEAMER. Además, desde un punto de vista matemático, aprenderemos el método de Interpolación Polinómica de Newton para la aproximación de una función en un intervalo determinado, haciendo uso de las diferencias divididas de Newton.
%---------------------------------------------------------------------------------
\section{Sección Uno: \LaTeX}
\label{1:sec:1}
 Es un sistema de composición muy adecuado para realizar documentos científicos y matemáticos de alta calidad tipográfica. Es también adecuado para producir documentos de cualquier otro tipo, desde simples cartas a libros enteros.\par\LaTeX{} está formado mayoritariamente por órdenes construidas a partir de comandos de \TeX{} (lenguaje de nivel bajo), en el sentido de que sus acciones son muy elementales, pero con la ventaja añadida de poder aumentar las capacidades de \LaTeX{} utilizando comandos propios del \TeX{} descritos en The TeXbook.3 4. Esto es lo que convierte a \LaTeX{} en una herramienta práctica y útil pues, a su facilidad de uso, se une toda la potencia de \TeX{}. Estas características hicieron que \LaTeX{} se extendiese rápidamente entre un amplio sector científico y técnico, hasta el punto de convertirse en uso obligado en comunicaciones y congresos, y requerido por determinadas revistas a la hora de entregar artículos académicos.
%---------------------------------------------------------------------------------
\section{Sección Dos: BEAMER}
\label{1:sec:2}
El nombre viene del vocablo alemán "beamer", un pseudo-anglicismo que significa videoproyector. BEAMER es una clase de \LaTeX{} para la creación de presentaciones. Funciona con pdflatex, dvips y LyX.\par Al estar basado en LaTeX, Beamer es especialmente útil para preparar presentaciones en las que es necesario mostrar gran cantidad de expresiones matemáticas, el fuerte de dicho sistema de maquetación.
