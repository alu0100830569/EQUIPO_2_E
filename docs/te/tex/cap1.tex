 %%%%%%%%%%%%%%%%%%%%%%%%%%%%%%%%%%%%%%%%%%%%%%%%%%%%%%%%%%%%%%%%%%%%%%%%%%%%%
% Chapter 1: Motivación y Objetivos
%%%%%%%%%%%%%%%%%%%%%%%%%%%%%%%%%%%%%%%%%%%%%%%%%%%%%%%%%%%%%%%%%%%%%%%%%%%%%%%

Los objetivos le dan al lector las razones por las que se realizó el
proyecto o trabajo de investigación.

%---------------------------------------------------------------------------------
\section{Sección Uno}
\label{1:sec:1}
  Primer párrafo de la primera sección.


%---------------------------------------------------------------------------------
\section{Sección Dos}
\label{1:sec:2}
  Primer párrafo de la segunda sección.

\begin{itemize}
  \item Item 1
  \item Item 2
  \item Item 3
\end{itemize}
 
 Si simplemente se desea escribir texto normal en LaTeX,
 sin complicadas fórmmulas matemáticas o efectos especiales
 como cambios de fuente, entonces simplemente tiene que escribir
 en español normalmente.\par
 Si se desea cambiar de párrafo ha de dejar una línea en blanco o bien 
 utilizar el comando. \par
 No es necesario preocuparse cde la sangría de los párrafos:
 todos los párrafos se sangrarán automáticamente con la expresión
 del primer párrafo de una sección.\par
 Se ha de distinguir entre la comilla simple 'izquierda'
 y la comilla simple 'derecha' cuando se escribe en el ordenador.\par
 En el caso de que se quiera utilizar comillas dobles se han de 
 escribir dos caracteres 'comilla simple' seguidos, esto es,
 ''comillas dobles''.\par
 También se ha de tener cuidado con los guiones: se utiliza un único
 guión para la separación de sílabas, mientras que se utilizan
 tres guines seguidos para producir un gruión de los que se usan
 como signo de puntuación --- como en esta oración.